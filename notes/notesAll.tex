\documentclass{article}

%=============================================
%===       importation des packages.        ==
%=============================================
\usepackage{tabularx, makecell, multirow }
\usepackage{syntonly}
%hpyerlien
\usepackage{hyperref}
%utf8
\usepackage[utf8]{inputenc}
%to solve footnote too high 
%(this helps fix footnote to bottom of the page)
\usepackage[bottom]{footmisc}
%array: aimed at tabular
\usepackage{array}
% sanxianbiao
\usepackage{booktabs}
% graphics
\usepackage{graphicx}
%math
\usepackage{amsmath}
\usepackage{amsthm}
%Algo
\usepackage{algorithm}
\usepackage{algorithmic}
%code
\usepackage{listings}
\usepackage{xcolor}
\usepackage[]{amssymb}

%=============================================
%===            Code config                 ==
%=============================================

%New colors defined below
\definecolor{codegreen}{rgb}{0,0.6,0}
\definecolor{codegray}{rgb}{0.5,0.5,0.5}
\definecolor{codepurple}{rgb}{0.58,0,0.82}
\definecolor{backcolour}{rgb}{0.95,0.95,0.92}

%Code listing style named "mystyle"
\lstdefinestyle{mystyle}{
  backgroundcolor=\color{backcolour}, commentstyle=\color{codegreen},
  keywordstyle=\color{magenta},
  numberstyle=\tiny\color{codegray},
  stringstyle=\color{codepurple},
  basicstyle=\ttfamily\footnotesize,
  breakatwhitespace=false,         
  breaklines=true,                 
  captionpos=b,                    
  keepspaces=true,                 
  numbers=left,                    
  numbersep=5pt,                  
  showspaces=false,                
  showstringspaces=false,
  showtabs=false,                  
  tabsize=2
}

%"mystyle" code listing set
%++++++++++++++++ end config code +++++++++++
\lstset{style=mystyle}

%=============================================
%===       Définition des environnements.   ==
%=============================================
\newtheorem{theoreme}{Théorème}
\newtheorem*{theoreme*}{Théorème}
\newtheorem{proposition}{Proposition}
\newtheorem*{proposition*}{Proposition}
\newtheorem{lemme}{Lemme}[section]
\newtheorem*{lemme*}{Lemme}
\newtheorem{corollaire}{Corollaire}
\newtheorem*{corollaire*}{Corollaire}
\newtheorem{conjecture}{Conjecture}
\newtheorem*{conjecture*}{Conjecture}

\theoremstyle{definition}
\newtheorem{definition}{Définition}
\newtheorem*{definition*}{Définition}
\newtheorem{notation}{Notation}
\newtheorem*{notation*}{Notation}
\newtheorem{exercice}{Exercice}
\newtheorem*{exercice*}{Exercice}
\newtheorem{probleme}{Problème}
\newtheorem*{probleme*}{Problème}
\newtheorem{question}{Question}
\newtheorem*{question*}{Question}
\newtheorem*{solution}{Solution}
\newtheorem*{code}{Code}

\theoremstyle{remark}
\newtheorem{exemple}{Exemple}[section]
\newtheorem*{exemple*}{Exemple}
\newtheorem{remarque}{Remarque}[section]
\newtheorem*{remarque*}{Remarque}
\newtheorem*{enigme}{Énigme}

%=============================================
%===       Définition des new commands.     ==
%=============================================
% Notation -- délimiteurs :
\newcommand{\ac}[1]{\left\{ #1 \right\}} 
% accolades
\newcommand{\pr}[1]{\left( #1 \right)} 
% parenthèses
\newcommand{\ct}[1]{\left[ #1 \right]} 
% crochets
\newcommand{\norm}[1]{\left\Vert #1 \right\Vert} 
% norme
\newcommand{\floor}[1]{\left\lfloor #1 \right\rfloor} 
% partie entière
\newcommand{\ceil}[1]{\left\lceil #1 \right\rceil} 
% partie entière + 1
\newcommand{\cond}{\ \middle\vert \ }


\title{\textbf{IFT2125 Notes}}
\author{Yuchen Hui 20150470}
\date{\today}


\begin{document}
    \maketitle
    \tableofcontents
    \section{Greedy Algorithm}
        \subsection{Sac à dos greedy version}
        \begin{proof}
            Supposons que les objets sont numerotes par ordre
            decroissant de valeur par unite de poids, i.e.
            $$\frac{v1}{w1} \geq\frac{v2}{w2}... $$
            par l'algorithme vorace. SI tous les $x_i = 1$,alors la solution
            est trivialement optimale.

            Sinon,soit j le plus petit indice tel que $x_j < 1$, on a 
            alors que $x_i = 1, \forall i <j$ et $x_i  = 0, \forall i >j$
            et $\sum_{i=1}^n x_iw_i = W$.

            Soit $V\pr{x}\sum_{i=1}^n x_iv_i $, la valeur de la soluttion X
            on doit demontrer que $V\pr{x}$ est maximale.

            Soit $Y = \pr{x_1,x_2,...,x_n}$  une autre solution de probleme et soit $V\pr{y}$ sa valeur.
            comme Y est une solution , $0 \leq y_i \leq 1 ,\forall i $et$ \sum_{i=1}^n y_iw_i = W$
            
            On veut montrer que $V\pr{x} - V\pr{y} \geq 0$ et alors X sera la solution optimale
            Soit j le plus petit indice tel que $x_j < 1$ 
            si i <j, alors $\frac{v_i}{w_i} \geq\frac{v_j}{w_j}  et x_i = 1$

        



        \end{proof}
    \subsection{File d'attente}
    Strategie vorace: classe les clients par ordre croissant des ti et executer les taches dans cet ordre
    \begin{proof}
        Soit n clients ordonne arbitrairement est servit selon l'ordre
        c = 1,2,3,..,n
        
        Le temps total de service requit est $T\pr{c} = t_1+ \pr{r+1+t_2}...+ = nt_1 + \pr{n-1}t_2 +...+ t_n$

        preuve par contradiction: i.e. qu<on suppose $T\pr{c}$ est optimal et c n'est pas l'ordre
        dans lequel on sert les clients en ordre croissant des $t_i$
    \end{proof}
    \subsection{Kruscal Algo}
    \subsubsection{Matrix Version Algo}
    nothing here


\section{Programmation dynamique}
\subsection{Knapsack Problem 2}
    \begin{algorithm} 
	\caption{\textbf{fonction} knapsack\_dy($w[1..n],v[1..n],W$): \textbf{array} V[0..n, 0..W]
} 
	\label{alg_sacados2} 
	\begin{algorithmic}
		\REQUIRE $v_{i} > 0, w_{i} > 0, x_{i} \in \left\{ 0,1 \right\}, W \in \mathbb{N}^{*} $ \\
		\COMMENT{array $w[1..n]$ indicates weights of objects 1 to n, array $n[1..n]$ indicates their values. $W$ is the max weight a knapsack can bear. Here come initialisations:}
		\textbf{array} $w[1..n] = $??? \\
		\textbf{array} $v[1..n] =$ ???\\ 
		\textbf{array} $V[0..n,0..W]$\\ 
		\textbf{for} $j = 1$ \textbf{to} $W$ \textbf{do} $V\left[ 0,j \right] = 0$\\
		\COMMENT{establish matrix}
		\FOR{$i = 1   $  \TO $n$}
		\FOR{$j = 1   $  \TO $W$}
		\STATE $V[i,j] \gets$ \textbf{if} $j-w[i]<0 $ \textbf{then} $V[i-1,j]$ 
		%\STATE \qquad\qquad\;\textbf{if} $i=1$ \textbf{then}  
		\STATE \qquad\qquad\;\textbf{else} $V\left[ i,j \right] = \text{max}\left( V[i-1,j], V[i-1,j-w[i]] + v[i] \right) $
		\ENDFOR
		\ENDFOR
		\RETURN $V[0..n,0..W]$
	\end{algorithmic} 
\end{algorithm}
\textbf{Et voici les codes:}
\begin{lstlisting}[language = Python, caption = codes pour knapsack2]
	
import time
import sys

# function from Marc feely
def createMatrix(numRow,numColumn):
    result = [None] * numRow
    for i in range(numRow):
        result[i] = [0]*numColumn
    return result


# 1. weight,value的index引用全部都-1,因为python的list的index从0开始。。。。
# 2. matrix include line 0 ( a line filled with 0), but the matrix
#    in manual don't have line 0 but begin with line 1
def knapsack_dynamic(weights, values, W_max, waitTime):
    num_objects = len(weights)
    V = createMatrix(len(weights)+1,W_max+1) 
    for j in range(1,W_max+1): V[0][j] = 0
    for i in range(1,num_objects+1):
        for j in range(0, W_max+1):
            if (j-weights[i-1] < 0): 
                V[i][j] = V[i-1][j]
            else: 
                V[i][j] = max(V[i-1][j],V[i-1][j-weights[i-1]]+values[i-1])
            
            print(V[i][j], "\t", end = "")
            sys.stdout.flush()
            time.sleep(waitTime)
        print("\n")

            
    print(V)

knapsack_dynamic([1,2,5,6,7],[1,6,18,22,28],11,1)
\end{lstlisting}

\section{Divide and Conquer}



    \end{document}
